\documentclass[12pt]{article}
\usepackage{mgates-letter}
\definecolor{dark_blue} {rgb}{0., 0., 0.65}

\usepackage{textcomp}
\usepackage{mathrsfs}  % mathscr font
\usepackage{boxedminipage}
\usepackage{rotating}
\usepackage[inline]{enumitem}
%\usepackage{natbib}
\usepackage[colorlinks, filecolor=dark_blue, urlcolor=dark_blue, linkcolor=black, citecolor=black]{hyperref}
\begin{document}

\begin{titlepage}

	\newcommand{\HRule}{\rule{\linewidth}{0.5mm}}
	\center
	
	\textsc{\Large Ph.D. Programme in Computer Science And Engineering}\\[0.5cm]
	
	\textsc{\Large XXXIX Cycle}\\[0.6cm]
	
	\hrule width \hsize \kern 1mm \hrule width \hsize height 2pt 
	\vspace{0.8cm}
	{ \large \bfseries Ph.D. Thesis Proposal}\\[0.6cm]
	{ \large Engineering Many-Agent Cooperative Learning in Collective Adaptive Systems }\\[0.6cm]


	% Key World: 
		% Cooperative Learning
		% Self-Adaptive Systems
		% Collective Adaptive Systems
		% Edge Cloud Continuum (?)
		% Collective Systems
		% Cooperative Intelligent Systems
		% Distributed Environments


	% Towards Cooperative Self-Adaptive Learning in Collective Intelligent Systems
	% Engineering Cooperative Learning in Collective Adaptive Systems

	\bfseries{February, 2024}


    \vspace{1.5cm}
    
    \noindent
    \begin{minipage}[t]{0.45\textwidth}
        \raggedright
        \textbf{Supervisors:}\\[0.5cm]
        Prof. Mirko Viroli\\
        Prof. Danilo Pianini\\
        Prof. Matteo Ferrara
    \end{minipage}%
    \hfill
    \begin{minipage}[t]{0.45\textwidth}
        \raggedleft
        \textbf{PhD Candidate:}\\[0.5cm]
        Davide Domini
    \end{minipage} \\[0.6cm]
	% {\bfseries{January, 2024}
	% \hfill
	% \bfseries{Davide Domini}}\\[0.6cm]
	
	% {\bfseries{Supervisors}: \\ [0.1cm]
	% Mirko Viroli \\
	% Danilo Pianini \\
	% Matteo Ferrara
	% } \\[0.6cm]

	\hrule width \hsize height 2pt \kern 1mm \hrule width \hsize height 1pt
	\vspace{0.4cm}
	%\begin{abstract}
	%\end{abstract}
	
\end{titlepage}

\section{Research Context}\label{sec:intro}

Computing devices have become ubiquitous in everyday life.
%
This trend has paved the way for research fields aimed at exploiting
 the potential of device collectives to build next-generation systems, 
 including: collective computing~\cite{DBLP:journals/computer/Abowd16}
 and collective adaptive systems (CAS)~\cite{DBLP:journals/sttt/WirsingJN23,robyphdthesis}.
%
More in particular, following Mitchell's definition~\cite{DBLP:conf/metacognition/Mitchell05}, 
 in this thesis we refer to CAS as distributed systems comprising multiple agents 
 such that each agent:
 \begin{enumerate*}[label=(\roman*)]
	\item can interact with other agents either directly or indirectly;
	\item does not individually posses system-wide knowledge;
	\item can exhibit learning to expand its personal knowledge; and
	\item can make decisions based on collective or aggregated knowledge from some of its peers.
 \end{enumerate*}

These systems enable the development of innovative applications in a wide range of real-world domains, such as: 
 smart cities~\cite{DBLP:conf/icse/IftikharRBW017}, 
 traffic control~\cite{DBLP:journals/tits/ChuWCL20,DBLP:books/sp/Muller2011/ProthmannTBHMS11} 
 with autonomous vehicles~\cite{DBLP:journals/corr/BojarskiTDFFGJM16}, 
 coordinated robot swarms for search and rescue~\cite{DBLP:journals/ijon/ZhouLLXS21} 
 or environmental monitoring~\cite{DBLP:conf/acsos/AguzziVE23}, and many more.
%



\section{Contribution}\label{sec:contribution}

\subsection{Methodological}

\subsection{Technological}


\section{Future Work}\label{sec:future}


\bibliographystyle{unsrt}
\bibliography{bibliography}

\end{document}
