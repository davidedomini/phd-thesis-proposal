\documentclass[12pt]{article}
\usepackage{mgates-letter}
\definecolor{dark_blue} {rgb}{0., 0., 0.65}

\usepackage{textcomp}
\usepackage{mathrsfs}  % mathscr font
\usepackage{boxedminipage}
\usepackage{rotating}
%\usepackage{natbib}
\usepackage[colorlinks, filecolor=dark_blue, urlcolor=dark_blue, linkcolor=black, citecolor=black]{hyperref}
\begin{document}

\begin{titlepage}

	\newcommand{\HRule}{\rule{\linewidth}{0.5mm}}
	\center
	
	\textsc{\Large Ph.D. Programme in Computer Science And Engineering}\\[0.5cm]
	
	\textsc{\Large XXXIX Cycle}\\[0.6cm]
	
	\hrule width \hsize \kern 1mm \hrule width \hsize height 2pt 
	\vspace{0.8cm}
	{ \large \bfseries Ph.D. Thesis Proposal}\\[0.6cm]
	{ \large Engineering Cooperative Learning in Collective Adaptive Systems}\\[0.6cm]
	
	
	% Towards AI-based macro-programming
	% Towards the integration of macro-programming and artificial intelligence
	% Hybrid AI-based approach for engineering cyber-physical swarms
	% A hybrid approach for engineering CPSW: toward the integration of macro-programming and Artificial Intelligence
	%An Hybrid Approach For Engineering Cyber-Physical Swarms: Aggregate Computing 
	%Enriching Collective Systems Engineering With Artificial Intelligence
	
	{\bfseries{January, 2024}
	\hfill
	\bfseries{Davide Domini}}\\[0.6cm]
	
	{\bfseries{Supervisors}: \\ [0.1cm]
	Mirko Viroli \\
	Danilo Pianini \\
	Matteo Ferrara
	} \\[0.6cm]

	\hrule width \hsize height 2pt \kern 1mm \hrule width \hsize height 1pt
	\vspace{0.4cm}
	%\begin{abstract}
	%\end{abstract}
	
\end{titlepage}

\section{Introduction}\label{sec:intro}

\section{Background and related work}\label{sec:related}

\section{Project Description}\label{sec:project}

\subsection{Activities}\label{sec:activities}

\subsection{Scope}\label{sec:scope}

\subsection{Technology}\label{sec:technology}

\section{Results}\label{sec:results}



\noindent
\nocite{*}

\bibliographystyle{plain}
\bibliography{bibliography}

\end{document}
