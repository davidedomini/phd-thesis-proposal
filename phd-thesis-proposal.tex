\documentclass[12pt]{article}
\usepackage{mgates-letter}
\definecolor{dark_blue} {rgb}{0., 0., 0.65}

\usepackage{textcomp}
\usepackage{mathrsfs}  % mathscr font
\usepackage{boxedminipage}
\usepackage{rotating}
\usepackage[inline]{enumitem}
%\usepackage{natbib}
\usepackage{xcolor}
\usepackage[colorlinks, filecolor=dark_blue, urlcolor=dark_blue, linkcolor=black, citecolor=black]{hyperref}

\newcommand{\todoinline}[1]{{\color{violet} #1}}

\begin{document}

\begin{titlepage}

	\newcommand{\HRule}{\rule{\linewidth}{0.5mm}}
	\center
	
	\textsc{\Large Ph.D. Programme in Computer Science And Engineering}\\[0.5cm]
	
	\textsc{\Large XXXIX Cycle}\\[0.6cm]
	
	\hrule width \hsize \kern 1mm \hrule width \hsize height 2pt 
	\vspace{0.8cm}
	{ \large \bfseries Ph.D. Thesis Proposal}\\[0.6cm]
	{ \large Engineering Many-Agent Cooperative Learning in Collective Adaptive Systems }\\[0.6cm]


	\bfseries{February, 2024}


    \vspace{1.5cm}
    
    \noindent
    \begin{minipage}[t]{0.45\textwidth}
        \raggedright
        \textbf{Supervisors:}\\[0.5cm]
        Prof. Mirko Viroli\\
        Prof. Danilo Pianini\\
        Prof. Matteo Ferrara
    \end{minipage}%
    \hfill
    \begin{minipage}[t]{0.45\textwidth}
        \raggedleft
        \textbf{PhD Candidate:}\\[0.5cm]
        Davide Domini
    \end{minipage} \\[0.6cm]

	\hrule width \hsize height 2pt \kern 1mm \hrule width \hsize height 1pt
	\vspace{0.4cm}

\end{titlepage}

\section{Research Context}\label{sec:intro}

Computing devices have become ubiquitous in everyday life.
%
This trend has paved the way for research fields aimed at exploiting
 the potential of device collectives to build next-generation systems, 
 including: collective computing~\cite{DBLP:journals/computer/Abowd16}
 and collective adaptive systems (CAS)~\cite{DBLP:journals/sttt/WirsingJN23,robyphdthesis}.
%
More in particular, following Mitchell's definition~\cite{DBLP:conf/metacognition/Mitchell05}, 
 in this thesis we refer to CAS as distributed systems comprising multiple agents 
 such that each agent: \todoinline{Say that agents number is very high, i.e. explain concept of many}
 \begin{enumerate*}[label=(\roman*)]
	\item can interact with other agents either directly or indirectly;
	\item does not individually posses system-wide knowledge;
	\item can exhibit learning to expand its personal knowledge; and
	\item can make decisions based on collective or aggregated knowledge from some of its peers.
 \end{enumerate*}

These systems enable the development of innovative applications in a wide range of real-world domains, such as: 
 smart cities~\cite{DBLP:conf/icse/IftikharRBW017}, 
 traffic control~\cite{DBLP:journals/tits/ChuWCL20,DBLP:books/sp/Muller2011/ProthmannTBHMS11} 
 with autonomous vehicles~\cite{DBLP:journals/corr/BojarskiTDFFGJM16}, 
 coordinated robot swarms for search and rescue~\cite{DBLP:journals/ijon/ZhouLLXS21} 
 or environmental monitoring~\cite{DBLP:conf/acsos/AguzziVE23}, and many more.
%
However, despite this great potential, several challenges arise when engineering these systems.
%
First and foremost, control and decision-making demand particular attention.
%
Achieving an optimal balance between centralized and decentralized control is crucial, 
 as neither extreme is feasible nor desirable in dynamic systems~\cite{DBLP:conf/coordination/CasadeiPVN19}. 
% 
Excessive centralization may lead to bottlenecks and single points of failure, 
 whereas complete decentralization can hinder coordination and consistency.
%
Additionally, the dynamic nature of these systems, characterized by constant 
 environmental changes, mobility, and potential component failures, requires adaptive 
 learning mechanisms capable of responding swiftly 
 to evolving conditions~\cite{DBLP:journals/swarm/PrasetyoMF19}.
%
Another key consideration is the locality principle, where operational efficiency and cost 
 are heavily influenced by the spatial proximity of data sources, processing units, and users.
%
Furthermore, partial observability introduces uncertainty, as individual components may have 
 limited or incomplete information about the global state, complicating accurate 
 decision-making~\cite{DBLP:conf/uai/HeDB22}.
%
Data privacy is also a growing concern, particularly in light of stringent regulations like GDPR~\cite{GDPR}
 in the European Union, necessitating privacy-preserving learning techniques.
%
Finally, data heterogeneity~\cite{DBLP:journals/fgcs/MaZLCQ22,DBLP:journals/ijon/ZhuXLJ21},
 stemming from diverse source, can significantly impact learning stability and accuracy.
%
Addressing these challenges is essential for the effective deployment of cooperative learning 
 in collective adaptive systems.



\section{Contribution}\label{sec:contribution}

\subsection{Methodological}

\subsection{Technological}


\section{Future Work}\label{sec:future}


\bibliographystyle{unsrt}
\bibliography{bibliography}

\end{document}
